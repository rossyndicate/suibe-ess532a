% Options for packages loaded elsewhere
\PassOptionsToPackage{unicode}{hyperref}
\PassOptionsToPackage{hyphens}{url}
%
\documentclass[
]{article}
\usepackage{amsmath,amssymb}
\usepackage{iftex}
\ifPDFTeX
  \usepackage[T1]{fontenc}
  \usepackage[utf8]{inputenc}
  \usepackage{textcomp} % provide euro and other symbols
\else % if luatex or xetex
  \usepackage{unicode-math} % this also loads fontspec
  \defaultfontfeatures{Scale=MatchLowercase}
  \defaultfontfeatures[\rmfamily]{Ligatures=TeX,Scale=1}
\fi
\usepackage{lmodern}
\ifPDFTeX\else
  % xetex/luatex font selection
\fi
% Use upquote if available, for straight quotes in verbatim environments
\IfFileExists{upquote.sty}{\usepackage{upquote}}{}
\IfFileExists{microtype.sty}{% use microtype if available
  \usepackage[]{microtype}
  \UseMicrotypeSet[protrusion]{basicmath} % disable protrusion for tt fonts
}{}
\makeatletter
\@ifundefined{KOMAClassName}{% if non-KOMA class
  \IfFileExists{parskip.sty}{%
    \usepackage{parskip}
  }{% else
    \setlength{\parindent}{0pt}
    \setlength{\parskip}{6pt plus 2pt minus 1pt}}
}{% if KOMA class
  \KOMAoptions{parskip=half}}
\makeatother
\usepackage{xcolor}
\usepackage[margin=1in]{geometry}
\usepackage{color}
\usepackage{fancyvrb}
\newcommand{\VerbBar}{|}
\newcommand{\VERB}{\Verb[commandchars=\\\{\}]}
\DefineVerbatimEnvironment{Highlighting}{Verbatim}{commandchars=\\\{\}}
% Add ',fontsize=\small' for more characters per line
\usepackage{framed}
\definecolor{shadecolor}{RGB}{248,248,248}
\newenvironment{Shaded}{\begin{snugshade}}{\end{snugshade}}
\newcommand{\AlertTok}[1]{\textcolor[rgb]{0.94,0.16,0.16}{#1}}
\newcommand{\AnnotationTok}[1]{\textcolor[rgb]{0.56,0.35,0.01}{\textbf{\textit{#1}}}}
\newcommand{\AttributeTok}[1]{\textcolor[rgb]{0.13,0.29,0.53}{#1}}
\newcommand{\BaseNTok}[1]{\textcolor[rgb]{0.00,0.00,0.81}{#1}}
\newcommand{\BuiltInTok}[1]{#1}
\newcommand{\CharTok}[1]{\textcolor[rgb]{0.31,0.60,0.02}{#1}}
\newcommand{\CommentTok}[1]{\textcolor[rgb]{0.56,0.35,0.01}{\textit{#1}}}
\newcommand{\CommentVarTok}[1]{\textcolor[rgb]{0.56,0.35,0.01}{\textbf{\textit{#1}}}}
\newcommand{\ConstantTok}[1]{\textcolor[rgb]{0.56,0.35,0.01}{#1}}
\newcommand{\ControlFlowTok}[1]{\textcolor[rgb]{0.13,0.29,0.53}{\textbf{#1}}}
\newcommand{\DataTypeTok}[1]{\textcolor[rgb]{0.13,0.29,0.53}{#1}}
\newcommand{\DecValTok}[1]{\textcolor[rgb]{0.00,0.00,0.81}{#1}}
\newcommand{\DocumentationTok}[1]{\textcolor[rgb]{0.56,0.35,0.01}{\textbf{\textit{#1}}}}
\newcommand{\ErrorTok}[1]{\textcolor[rgb]{0.64,0.00,0.00}{\textbf{#1}}}
\newcommand{\ExtensionTok}[1]{#1}
\newcommand{\FloatTok}[1]{\textcolor[rgb]{0.00,0.00,0.81}{#1}}
\newcommand{\FunctionTok}[1]{\textcolor[rgb]{0.13,0.29,0.53}{\textbf{#1}}}
\newcommand{\ImportTok}[1]{#1}
\newcommand{\InformationTok}[1]{\textcolor[rgb]{0.56,0.35,0.01}{\textbf{\textit{#1}}}}
\newcommand{\KeywordTok}[1]{\textcolor[rgb]{0.13,0.29,0.53}{\textbf{#1}}}
\newcommand{\NormalTok}[1]{#1}
\newcommand{\OperatorTok}[1]{\textcolor[rgb]{0.81,0.36,0.00}{\textbf{#1}}}
\newcommand{\OtherTok}[1]{\textcolor[rgb]{0.56,0.35,0.01}{#1}}
\newcommand{\PreprocessorTok}[1]{\textcolor[rgb]{0.56,0.35,0.01}{\textit{#1}}}
\newcommand{\RegionMarkerTok}[1]{#1}
\newcommand{\SpecialCharTok}[1]{\textcolor[rgb]{0.81,0.36,0.00}{\textbf{#1}}}
\newcommand{\SpecialStringTok}[1]{\textcolor[rgb]{0.31,0.60,0.02}{#1}}
\newcommand{\StringTok}[1]{\textcolor[rgb]{0.31,0.60,0.02}{#1}}
\newcommand{\VariableTok}[1]{\textcolor[rgb]{0.00,0.00,0.00}{#1}}
\newcommand{\VerbatimStringTok}[1]{\textcolor[rgb]{0.31,0.60,0.02}{#1}}
\newcommand{\WarningTok}[1]{\textcolor[rgb]{0.56,0.35,0.01}{\textbf{\textit{#1}}}}
\usepackage{graphicx}
\makeatletter
\newsavebox\pandoc@box
\newcommand*\pandocbounded[1]{% scales image to fit in text height/width
  \sbox\pandoc@box{#1}%
  \Gscale@div\@tempa{\textheight}{\dimexpr\ht\pandoc@box+\dp\pandoc@box\relax}%
  \Gscale@div\@tempb{\linewidth}{\wd\pandoc@box}%
  \ifdim\@tempb\p@<\@tempa\p@\let\@tempa\@tempb\fi% select the smaller of both
  \ifdim\@tempa\p@<\p@\scalebox{\@tempa}{\usebox\pandoc@box}%
  \else\usebox{\pandoc@box}%
  \fi%
}
% Set default figure placement to htbp
\def\fps@figure{htbp}
\makeatother
\setlength{\emergencystretch}{3em} % prevent overfull lines
\providecommand{\tightlist}{%
  \setlength{\itemsep}{0pt}\setlength{\parskip}{0pt}}
\setcounter{secnumdepth}{-\maxdimen} % remove section numbering
\usepackage{bookmark}
\IfFileExists{xurl.sty}{\usepackage{xurl}}{} % add URL line breaks if available
\urlstyle{same}
\hypersetup{
  pdftitle={Data Exploration Assignment: Bison Dataset},
  pdfauthor={Katie Willi},
  hidelinks,
  pdfcreator={LaTeX via pandoc}}

\title{Data Exploration Assignment: Bison Dataset}
\author{Katie Willi}
\date{2025-06-03}

\begin{document}
\maketitle

\subsection{Introduction}\label{introduction}

In this assignment, we will be using a new dataset containing
information collected about bison on the Konza Prairie in Kansas, USA.

\textbf{Data Credit:} EBlair, J. 2021. CBH01 Konza Prairie bison herd
information ver 12. Environmental Data Initiative.
\url{https://doi.org/10.6073/pasta/9c641b35695abc5889edd64c3950517f}
(Accessed 2021-05-10).
\url{doi:10.6073/pasta/9c641b35695abc5889edd64c3950517f}

\textbf{Dataset Variables:} - \texttt{data\_code}: a character denoting
the dataset code - \texttt{rec\_year}: a number denoting the year of
observation - \texttt{rec\_month}: a number denoting the month of
observation - \texttt{rec\_day}: a number denoting the day of
observation - \texttt{animal\_code}: a character denoting the unique
individual bison identification code based on ear tag number -
\texttt{animal\_sex}: a character denoting the sex of bison: M = male, F
= female, U = unknown - \texttt{animal\_weight}: a number denoting bison
weight in pounds - \texttt{animal\_yob}: a number denoting the year
animal was born

\subsection{Setup}\label{setup}

First, load the necessary packages and read in the bison dataset:

\begin{Shaded}
\begin{Highlighting}[]
\FunctionTok{library}\NormalTok{(tidyverse)}

\CommentTok{\# Read in the bison dataset}
\NormalTok{bison }\OtherTok{\textless{}{-}} \FunctionTok{read\_csv}\NormalTok{(}\StringTok{"data\_exploration/data/bison\_data.csv"}\NormalTok{)}
\end{Highlighting}
\end{Shaded}

\subsection{Exercise 1: Basic Data
Exploration}\label{exercise-1-basic-data-exploration}

Use the following functions to explore the bison dataset:

\begin{enumerate}
\def\labelenumi{\alph{enumi})}
\tightlist
\item
  Use \texttt{nrow()} and \texttt{ncol()} to find how many rows and
  columns the dataset has
\item
  Use \texttt{names()} to see the column names
\item
  Use \texttt{summary()} to get a basic summary of all variables
\end{enumerate}

\begin{Shaded}
\begin{Highlighting}[]
\FunctionTok{nrow}\NormalTok{(bison)}
\end{Highlighting}
\end{Shaded}

\begin{verbatim}
## [1] 8325
\end{verbatim}

\begin{Shaded}
\begin{Highlighting}[]
\FunctionTok{ncol}\NormalTok{(bison)}
\end{Highlighting}
\end{Shaded}

\begin{verbatim}
## [1] 8
\end{verbatim}

\begin{Shaded}
\begin{Highlighting}[]
\FunctionTok{names}\NormalTok{(bison)}
\end{Highlighting}
\end{Shaded}

\begin{verbatim}
## [1] "data_code"     "rec_year"      "rec_month"     "rec_day"      
## [5] "animal_code"   "animal_sex"    "animal_weight" "animal_yob"
\end{verbatim}

\begin{Shaded}
\begin{Highlighting}[]
\FunctionTok{summary}\NormalTok{(bison)}
\end{Highlighting}
\end{Shaded}

\begin{verbatim}
##   data_code            rec_year      rec_month        rec_day     
##  Length:8325        Min.   :1994   Min.   :10.00   Min.   : 5.00  
##  Class :character   1st Qu.:2003   1st Qu.:10.00   1st Qu.:12.00  
##  Mode  :character   Median :2008   Median :10.00   Median :24.00  
##                     Mean   :2008   Mean   :10.48   Mean   :19.14  
##                     3rd Qu.:2015   3rd Qu.:11.00   3rd Qu.:27.00  
##                     Max.   :2020   Max.   :11.00   Max.   :31.00  
##                                                                   
##  animal_code         animal_sex        animal_weight      animal_yob  
##  Length:8325        Length:8325        Min.   :  90.0   Min.   :1981  
##  Class :character   Class :character   1st Qu.: 454.0   1st Qu.:1999  
##  Mode  :character   Mode  :character   Median : 777.0   Median :2005  
##                                        Mean   : 749.5   Mean   :2005  
##                                        3rd Qu.: 986.0   3rd Qu.:2011  
##                                        Max.   :2050.0   Max.   :2020  
##                                        NA's   :1        NA's   :251
\end{verbatim}

\subsection{Exercise 2: Simple
Filtering}\label{exercise-2-simple-filtering}

Filter the bison dataset to include only female bison (``F''). Store
this in a new object called \texttt{female\_bison}.

\begin{Shaded}
\begin{Highlighting}[]
\NormalTok{female\_bison }\OtherTok{\textless{}{-}}\NormalTok{ bison }\SpecialCharTok{\%\textgreater{}\%}
  \FunctionTok{filter}\NormalTok{(animal\_sex }\SpecialCharTok{==} \StringTok{"F"}\NormalTok{)}
\end{Highlighting}
\end{Shaded}

\subsection{Exercise 3: Creating a New
Variable}\label{exercise-3-creating-a-new-variable}

Use \texttt{mutate()} to create a new column in \texttt{bison} called
\texttt{age\_at\_recording} that calculates how old each bison was when
recorded.

\begin{Shaded}
\begin{Highlighting}[]
\NormalTok{bison }\OtherTok{\textless{}{-}}\NormalTok{ bison }\SpecialCharTok{\%\textgreater{}\%}
  \FunctionTok{mutate}\NormalTok{(}\AttributeTok{age\_at\_recording =}\NormalTok{ rec\_year }\SpecialCharTok{{-}}\NormalTok{ animal\_yob)}
\end{Highlighting}
\end{Shaded}

\subsection{Exercise 4: Using if\_else() for
Categories}\label{exercise-4-using-if_else-for-categories}

Create a new column in bison called \texttt{weight\_category} using
\texttt{if\_else()}. Classify bison as: - ``Heavy'' if their weight is
greater than 800 pounds - ``Light'' if their weight is 800 pounds or
less

\begin{Shaded}
\begin{Highlighting}[]
\NormalTok{bison }\OtherTok{\textless{}{-}}\NormalTok{ bison }\SpecialCharTok{\%\textgreater{}\%}
  \FunctionTok{mutate}\NormalTok{(}\AttributeTok{weight\_category =} \FunctionTok{if\_else}\NormalTok{(animal\_weight }\SpecialCharTok{\textgreater{}} \DecValTok{800}\NormalTok{, }\StringTok{"Heavy"}\NormalTok{, }\StringTok{"Light"}\NormalTok{))}
\end{Highlighting}
\end{Shaded}

\subsection{Exercise 5: Basic Grouping and
Summarizing}\label{exercise-5-basic-grouping-and-summarizing}

Find the average weight for each sex (M, F, U).

\begin{Shaded}
\begin{Highlighting}[]
\NormalTok{avg\_weight }\OtherTok{\textless{}{-}}\NormalTok{ bison }\SpecialCharTok{\%\textgreater{}\%}
  \FunctionTok{group\_by}\NormalTok{(animal\_sex) }\SpecialCharTok{\%\textgreater{}\%}
  \FunctionTok{summarize}\NormalTok{(}\AttributeTok{avg\_weight =} \FunctionTok{mean}\NormalTok{(animal\_weight, }\AttributeTok{na.rm =} \ConstantTok{TRUE}\NormalTok{))}
\end{Highlighting}
\end{Shaded}

\subsection{Exercise 6: Selecting and
Arranging}\label{exercise-6-selecting-and-arranging}

Select only the columns: animal\_code, animal\_sex, animal\_weight, and
rec\_year. Then arrange the data by weight from heaviest to lightest
using \texttt{arrange()}.

Hint: Look at the examples in \texttt{?arrange}.

\begin{Shaded}
\begin{Highlighting}[]
\NormalTok{subset\_bison }\OtherTok{\textless{}{-}}\NormalTok{ bison }\SpecialCharTok{\%\textgreater{}\%}
  \FunctionTok{select}\NormalTok{(animal\_code, animal\_sex, animal\_weight, rec\_year) }\SpecialCharTok{\%\textgreater{}\%}
  \FunctionTok{arrange}\NormalTok{(}\FunctionTok{desc}\NormalTok{(animal\_weight))}
\end{Highlighting}
\end{Shaded}

\subsection{Exercise 7: Basic Scatter
Plot}\label{exercise-7-basic-scatter-plot}

Create a scatter plot using \texttt{geom\_point()} to show the
relationship between recording year (x-axis) and bison weight (y-axis):
- Color the points by sex - Add appropriate axis labels

\begin{Shaded}
\begin{Highlighting}[]
\FunctionTok{ggplot}\NormalTok{(bison) }\SpecialCharTok{+}
  \FunctionTok{geom\_point}\NormalTok{(}\FunctionTok{aes}\NormalTok{(}\AttributeTok{x =}\NormalTok{ rec\_year, }\AttributeTok{y =}\NormalTok{ animal\_weight, }\AttributeTok{color =}\NormalTok{ animal\_sex)) }\SpecialCharTok{+}
  \FunctionTok{ylab}\NormalTok{(}\StringTok{"Animal Weight"}\NormalTok{) }\SpecialCharTok{+}
  \FunctionTok{xlab}\NormalTok{(}\StringTok{"Record Year"}\NormalTok{)}
\end{Highlighting}
\end{Shaded}

\pandocbounded{\includegraphics[keepaspectratio]{KEY_ASSIGNMENT_2_Data_Exploration_files/figure-latex/unnamed-chunk-8-1.pdf}}

\subsection{Bonus Question}\label{bonus-question}

Look at your scatter plot from Exercise 8. Do you notice any patterns?
Write 1-2 sentences about what you observe.

\textbf{Your observation:} {[}Write what you notice here{]}

\end{document}
